
% Default to the notebook output style

    


% Inherit from the specified cell style.




    
\documentclass[11pt]{article}

    
    
    \usepackage[T1]{fontenc}
    % Nicer default font (+ math font) than Computer Modern for most use cases
    \usepackage{mathpazo}

    % Basic figure setup, for now with no caption control since it's done
    % automatically by Pandoc (which extracts ![](path) syntax from Markdown).
    \usepackage{graphicx}
    % We will generate all images so they have a width \maxwidth. This means
    % that they will get their normal width if they fit onto the page, but
    % are scaled down if they would overflow the margins.
    \makeatletter
    \def\maxwidth{\ifdim\Gin@nat@width>\linewidth\linewidth
    \else\Gin@nat@width\fi}
    \makeatother
    \let\Oldincludegraphics\includegraphics
    % Set max figure width to be 80% of text width, for now hardcoded.
    \renewcommand{\includegraphics}[1]{\Oldincludegraphics[width=.8\maxwidth]{#1}}
    % Ensure that by default, figures have no caption (until we provide a
    % proper Figure object with a Caption API and a way to capture that
    % in the conversion process - todo).
    \usepackage{caption}
    \DeclareCaptionLabelFormat{nolabel}{}
    \captionsetup{labelformat=nolabel}

    \usepackage{adjustbox} % Used to constrain images to a maximum size 
    \usepackage{xcolor} % Allow colors to be defined
    \usepackage{enumerate} % Needed for markdown enumerations to work
    \usepackage{geometry} % Used to adjust the document margins
    \usepackage{amsmath} % Equations
    \usepackage{amssymb} % Equations
    \usepackage{textcomp} % defines textquotesingle
    % Hack from http://tex.stackexchange.com/a/47451/13684:
    \AtBeginDocument{%
        \def\PYZsq{\textquotesingle}% Upright quotes in Pygmentized code
    }
    \usepackage{upquote} % Upright quotes for verbatim code
    \usepackage{eurosym} % defines \euro
    \usepackage[mathletters]{ucs} % Extended unicode (utf-8) support
    \usepackage[utf8x]{inputenc} % Allow utf-8 characters in the tex document
    \usepackage{fancyvrb} % verbatim replacement that allows latex
    \usepackage{grffile} % extends the file name processing of package graphics 
                         % to support a larger range 
    % The hyperref package gives us a pdf with properly built
    % internal navigation ('pdf bookmarks' for the table of contents,
    % internal cross-reference links, web links for URLs, etc.)
    \usepackage{hyperref}
    \usepackage{longtable} % longtable support required by pandoc >1.10
    \usepackage{booktabs}  % table support for pandoc > 1.12.2
    \usepackage[inline]{enumitem} % IRkernel/repr support (it uses the enumerate* environment)
    \usepackage[normalem]{ulem} % ulem is needed to support strikethroughs (\sout)
                                % normalem makes italics be italics, not underlines
    

    
    
    % Colors for the hyperref package
    \definecolor{urlcolor}{rgb}{0,.145,.698}
    \definecolor{linkcolor}{rgb}{.71,0.21,0.01}
    \definecolor{citecolor}{rgb}{.12,.54,.11}

    % ANSI colors
    \definecolor{ansi-black}{HTML}{3E424D}
    \definecolor{ansi-black-intense}{HTML}{282C36}
    \definecolor{ansi-red}{HTML}{E75C58}
    \definecolor{ansi-red-intense}{HTML}{B22B31}
    \definecolor{ansi-green}{HTML}{00A250}
    \definecolor{ansi-green-intense}{HTML}{007427}
    \definecolor{ansi-yellow}{HTML}{DDB62B}
    \definecolor{ansi-yellow-intense}{HTML}{B27D12}
    \definecolor{ansi-blue}{HTML}{208FFB}
    \definecolor{ansi-blue-intense}{HTML}{0065CA}
    \definecolor{ansi-magenta}{HTML}{D160C4}
    \definecolor{ansi-magenta-intense}{HTML}{A03196}
    \definecolor{ansi-cyan}{HTML}{60C6C8}
    \definecolor{ansi-cyan-intense}{HTML}{258F8F}
    \definecolor{ansi-white}{HTML}{C5C1B4}
    \definecolor{ansi-white-intense}{HTML}{A1A6B2}

    % commands and environments needed by pandoc snippets
    % extracted from the output of `pandoc -s`
    \providecommand{\tightlist}{%
      \setlength{\itemsep}{0pt}\setlength{\parskip}{0pt}}
    \DefineVerbatimEnvironment{Highlighting}{Verbatim}{commandchars=\\\{\}}
    % Add ',fontsize=\small' for more characters per line
    \newenvironment{Shaded}{}{}
    \newcommand{\KeywordTok}[1]{\textcolor[rgb]{0.00,0.44,0.13}{\textbf{{#1}}}}
    \newcommand{\DataTypeTok}[1]{\textcolor[rgb]{0.56,0.13,0.00}{{#1}}}
    \newcommand{\DecValTok}[1]{\textcolor[rgb]{0.25,0.63,0.44}{{#1}}}
    \newcommand{\BaseNTok}[1]{\textcolor[rgb]{0.25,0.63,0.44}{{#1}}}
    \newcommand{\FloatTok}[1]{\textcolor[rgb]{0.25,0.63,0.44}{{#1}}}
    \newcommand{\CharTok}[1]{\textcolor[rgb]{0.25,0.44,0.63}{{#1}}}
    \newcommand{\StringTok}[1]{\textcolor[rgb]{0.25,0.44,0.63}{{#1}}}
    \newcommand{\CommentTok}[1]{\textcolor[rgb]{0.38,0.63,0.69}{\textit{{#1}}}}
    \newcommand{\OtherTok}[1]{\textcolor[rgb]{0.00,0.44,0.13}{{#1}}}
    \newcommand{\AlertTok}[1]{\textcolor[rgb]{1.00,0.00,0.00}{\textbf{{#1}}}}
    \newcommand{\FunctionTok}[1]{\textcolor[rgb]{0.02,0.16,0.49}{{#1}}}
    \newcommand{\RegionMarkerTok}[1]{{#1}}
    \newcommand{\ErrorTok}[1]{\textcolor[rgb]{1.00,0.00,0.00}{\textbf{{#1}}}}
    \newcommand{\NormalTok}[1]{{#1}}
    
    % Additional commands for more recent versions of Pandoc
    \newcommand{\ConstantTok}[1]{\textcolor[rgb]{0.53,0.00,0.00}{{#1}}}
    \newcommand{\SpecialCharTok}[1]{\textcolor[rgb]{0.25,0.44,0.63}{{#1}}}
    \newcommand{\VerbatimStringTok}[1]{\textcolor[rgb]{0.25,0.44,0.63}{{#1}}}
    \newcommand{\SpecialStringTok}[1]{\textcolor[rgb]{0.73,0.40,0.53}{{#1}}}
    \newcommand{\ImportTok}[1]{{#1}}
    \newcommand{\DocumentationTok}[1]{\textcolor[rgb]{0.73,0.13,0.13}{\textit{{#1}}}}
    \newcommand{\AnnotationTok}[1]{\textcolor[rgb]{0.38,0.63,0.69}{\textbf{\textit{{#1}}}}}
    \newcommand{\CommentVarTok}[1]{\textcolor[rgb]{0.38,0.63,0.69}{\textbf{\textit{{#1}}}}}
    \newcommand{\VariableTok}[1]{\textcolor[rgb]{0.10,0.09,0.49}{{#1}}}
    \newcommand{\ControlFlowTok}[1]{\textcolor[rgb]{0.00,0.44,0.13}{\textbf{{#1}}}}
    \newcommand{\OperatorTok}[1]{\textcolor[rgb]{0.40,0.40,0.40}{{#1}}}
    \newcommand{\BuiltInTok}[1]{{#1}}
    \newcommand{\ExtensionTok}[1]{{#1}}
    \newcommand{\PreprocessorTok}[1]{\textcolor[rgb]{0.74,0.48,0.00}{{#1}}}
    \newcommand{\AttributeTok}[1]{\textcolor[rgb]{0.49,0.56,0.16}{{#1}}}
    \newcommand{\InformationTok}[1]{\textcolor[rgb]{0.38,0.63,0.69}{\textbf{\textit{{#1}}}}}
    \newcommand{\WarningTok}[1]{\textcolor[rgb]{0.38,0.63,0.69}{\textbf{\textit{{#1}}}}}
    
    
    % Define a nice break command that doesn't care if a line doesn't already
    % exist.
    \def\br{\hspace*{\fill} \\* }
    % Math Jax compatability definitions
    \def\gt{>}
    \def\lt{<}
    % Document parameters
    \title{01-Object Oriented Programming}
    
    
    

    % Pygments definitions
    
\makeatletter
\def\PY@reset{\let\PY@it=\relax \let\PY@bf=\relax%
    \let\PY@ul=\relax \let\PY@tc=\relax%
    \let\PY@bc=\relax \let\PY@ff=\relax}
\def\PY@tok#1{\csname PY@tok@#1\endcsname}
\def\PY@toks#1+{\ifx\relax#1\empty\else%
    \PY@tok{#1}\expandafter\PY@toks\fi}
\def\PY@do#1{\PY@bc{\PY@tc{\PY@ul{%
    \PY@it{\PY@bf{\PY@ff{#1}}}}}}}
\def\PY#1#2{\PY@reset\PY@toks#1+\relax+\PY@do{#2}}

\expandafter\def\csname PY@tok@w\endcsname{\def\PY@tc##1{\textcolor[rgb]{0.73,0.73,0.73}{##1}}}
\expandafter\def\csname PY@tok@c\endcsname{\let\PY@it=\textit\def\PY@tc##1{\textcolor[rgb]{0.25,0.50,0.50}{##1}}}
\expandafter\def\csname PY@tok@cp\endcsname{\def\PY@tc##1{\textcolor[rgb]{0.74,0.48,0.00}{##1}}}
\expandafter\def\csname PY@tok@k\endcsname{\let\PY@bf=\textbf\def\PY@tc##1{\textcolor[rgb]{0.00,0.50,0.00}{##1}}}
\expandafter\def\csname PY@tok@kp\endcsname{\def\PY@tc##1{\textcolor[rgb]{0.00,0.50,0.00}{##1}}}
\expandafter\def\csname PY@tok@kt\endcsname{\def\PY@tc##1{\textcolor[rgb]{0.69,0.00,0.25}{##1}}}
\expandafter\def\csname PY@tok@o\endcsname{\def\PY@tc##1{\textcolor[rgb]{0.40,0.40,0.40}{##1}}}
\expandafter\def\csname PY@tok@ow\endcsname{\let\PY@bf=\textbf\def\PY@tc##1{\textcolor[rgb]{0.67,0.13,1.00}{##1}}}
\expandafter\def\csname PY@tok@nb\endcsname{\def\PY@tc##1{\textcolor[rgb]{0.00,0.50,0.00}{##1}}}
\expandafter\def\csname PY@tok@nf\endcsname{\def\PY@tc##1{\textcolor[rgb]{0.00,0.00,1.00}{##1}}}
\expandafter\def\csname PY@tok@nc\endcsname{\let\PY@bf=\textbf\def\PY@tc##1{\textcolor[rgb]{0.00,0.00,1.00}{##1}}}
\expandafter\def\csname PY@tok@nn\endcsname{\let\PY@bf=\textbf\def\PY@tc##1{\textcolor[rgb]{0.00,0.00,1.00}{##1}}}
\expandafter\def\csname PY@tok@ne\endcsname{\let\PY@bf=\textbf\def\PY@tc##1{\textcolor[rgb]{0.82,0.25,0.23}{##1}}}
\expandafter\def\csname PY@tok@nv\endcsname{\def\PY@tc##1{\textcolor[rgb]{0.10,0.09,0.49}{##1}}}
\expandafter\def\csname PY@tok@no\endcsname{\def\PY@tc##1{\textcolor[rgb]{0.53,0.00,0.00}{##1}}}
\expandafter\def\csname PY@tok@nl\endcsname{\def\PY@tc##1{\textcolor[rgb]{0.63,0.63,0.00}{##1}}}
\expandafter\def\csname PY@tok@ni\endcsname{\let\PY@bf=\textbf\def\PY@tc##1{\textcolor[rgb]{0.60,0.60,0.60}{##1}}}
\expandafter\def\csname PY@tok@na\endcsname{\def\PY@tc##1{\textcolor[rgb]{0.49,0.56,0.16}{##1}}}
\expandafter\def\csname PY@tok@nt\endcsname{\let\PY@bf=\textbf\def\PY@tc##1{\textcolor[rgb]{0.00,0.50,0.00}{##1}}}
\expandafter\def\csname PY@tok@nd\endcsname{\def\PY@tc##1{\textcolor[rgb]{0.67,0.13,1.00}{##1}}}
\expandafter\def\csname PY@tok@s\endcsname{\def\PY@tc##1{\textcolor[rgb]{0.73,0.13,0.13}{##1}}}
\expandafter\def\csname PY@tok@sd\endcsname{\let\PY@it=\textit\def\PY@tc##1{\textcolor[rgb]{0.73,0.13,0.13}{##1}}}
\expandafter\def\csname PY@tok@si\endcsname{\let\PY@bf=\textbf\def\PY@tc##1{\textcolor[rgb]{0.73,0.40,0.53}{##1}}}
\expandafter\def\csname PY@tok@se\endcsname{\let\PY@bf=\textbf\def\PY@tc##1{\textcolor[rgb]{0.73,0.40,0.13}{##1}}}
\expandafter\def\csname PY@tok@sr\endcsname{\def\PY@tc##1{\textcolor[rgb]{0.73,0.40,0.53}{##1}}}
\expandafter\def\csname PY@tok@ss\endcsname{\def\PY@tc##1{\textcolor[rgb]{0.10,0.09,0.49}{##1}}}
\expandafter\def\csname PY@tok@sx\endcsname{\def\PY@tc##1{\textcolor[rgb]{0.00,0.50,0.00}{##1}}}
\expandafter\def\csname PY@tok@m\endcsname{\def\PY@tc##1{\textcolor[rgb]{0.40,0.40,0.40}{##1}}}
\expandafter\def\csname PY@tok@gh\endcsname{\let\PY@bf=\textbf\def\PY@tc##1{\textcolor[rgb]{0.00,0.00,0.50}{##1}}}
\expandafter\def\csname PY@tok@gu\endcsname{\let\PY@bf=\textbf\def\PY@tc##1{\textcolor[rgb]{0.50,0.00,0.50}{##1}}}
\expandafter\def\csname PY@tok@gd\endcsname{\def\PY@tc##1{\textcolor[rgb]{0.63,0.00,0.00}{##1}}}
\expandafter\def\csname PY@tok@gi\endcsname{\def\PY@tc##1{\textcolor[rgb]{0.00,0.63,0.00}{##1}}}
\expandafter\def\csname PY@tok@gr\endcsname{\def\PY@tc##1{\textcolor[rgb]{1.00,0.00,0.00}{##1}}}
\expandafter\def\csname PY@tok@ge\endcsname{\let\PY@it=\textit}
\expandafter\def\csname PY@tok@gs\endcsname{\let\PY@bf=\textbf}
\expandafter\def\csname PY@tok@gp\endcsname{\let\PY@bf=\textbf\def\PY@tc##1{\textcolor[rgb]{0.00,0.00,0.50}{##1}}}
\expandafter\def\csname PY@tok@go\endcsname{\def\PY@tc##1{\textcolor[rgb]{0.53,0.53,0.53}{##1}}}
\expandafter\def\csname PY@tok@gt\endcsname{\def\PY@tc##1{\textcolor[rgb]{0.00,0.27,0.87}{##1}}}
\expandafter\def\csname PY@tok@err\endcsname{\def\PY@bc##1{\setlength{\fboxsep}{0pt}\fcolorbox[rgb]{1.00,0.00,0.00}{1,1,1}{\strut ##1}}}
\expandafter\def\csname PY@tok@kc\endcsname{\let\PY@bf=\textbf\def\PY@tc##1{\textcolor[rgb]{0.00,0.50,0.00}{##1}}}
\expandafter\def\csname PY@tok@kd\endcsname{\let\PY@bf=\textbf\def\PY@tc##1{\textcolor[rgb]{0.00,0.50,0.00}{##1}}}
\expandafter\def\csname PY@tok@kn\endcsname{\let\PY@bf=\textbf\def\PY@tc##1{\textcolor[rgb]{0.00,0.50,0.00}{##1}}}
\expandafter\def\csname PY@tok@kr\endcsname{\let\PY@bf=\textbf\def\PY@tc##1{\textcolor[rgb]{0.00,0.50,0.00}{##1}}}
\expandafter\def\csname PY@tok@bp\endcsname{\def\PY@tc##1{\textcolor[rgb]{0.00,0.50,0.00}{##1}}}
\expandafter\def\csname PY@tok@fm\endcsname{\def\PY@tc##1{\textcolor[rgb]{0.00,0.00,1.00}{##1}}}
\expandafter\def\csname PY@tok@vc\endcsname{\def\PY@tc##1{\textcolor[rgb]{0.10,0.09,0.49}{##1}}}
\expandafter\def\csname PY@tok@vg\endcsname{\def\PY@tc##1{\textcolor[rgb]{0.10,0.09,0.49}{##1}}}
\expandafter\def\csname PY@tok@vi\endcsname{\def\PY@tc##1{\textcolor[rgb]{0.10,0.09,0.49}{##1}}}
\expandafter\def\csname PY@tok@vm\endcsname{\def\PY@tc##1{\textcolor[rgb]{0.10,0.09,0.49}{##1}}}
\expandafter\def\csname PY@tok@sa\endcsname{\def\PY@tc##1{\textcolor[rgb]{0.73,0.13,0.13}{##1}}}
\expandafter\def\csname PY@tok@sb\endcsname{\def\PY@tc##1{\textcolor[rgb]{0.73,0.13,0.13}{##1}}}
\expandafter\def\csname PY@tok@sc\endcsname{\def\PY@tc##1{\textcolor[rgb]{0.73,0.13,0.13}{##1}}}
\expandafter\def\csname PY@tok@dl\endcsname{\def\PY@tc##1{\textcolor[rgb]{0.73,0.13,0.13}{##1}}}
\expandafter\def\csname PY@tok@s2\endcsname{\def\PY@tc##1{\textcolor[rgb]{0.73,0.13,0.13}{##1}}}
\expandafter\def\csname PY@tok@sh\endcsname{\def\PY@tc##1{\textcolor[rgb]{0.73,0.13,0.13}{##1}}}
\expandafter\def\csname PY@tok@s1\endcsname{\def\PY@tc##1{\textcolor[rgb]{0.73,0.13,0.13}{##1}}}
\expandafter\def\csname PY@tok@mb\endcsname{\def\PY@tc##1{\textcolor[rgb]{0.40,0.40,0.40}{##1}}}
\expandafter\def\csname PY@tok@mf\endcsname{\def\PY@tc##1{\textcolor[rgb]{0.40,0.40,0.40}{##1}}}
\expandafter\def\csname PY@tok@mh\endcsname{\def\PY@tc##1{\textcolor[rgb]{0.40,0.40,0.40}{##1}}}
\expandafter\def\csname PY@tok@mi\endcsname{\def\PY@tc##1{\textcolor[rgb]{0.40,0.40,0.40}{##1}}}
\expandafter\def\csname PY@tok@il\endcsname{\def\PY@tc##1{\textcolor[rgb]{0.40,0.40,0.40}{##1}}}
\expandafter\def\csname PY@tok@mo\endcsname{\def\PY@tc##1{\textcolor[rgb]{0.40,0.40,0.40}{##1}}}
\expandafter\def\csname PY@tok@ch\endcsname{\let\PY@it=\textit\def\PY@tc##1{\textcolor[rgb]{0.25,0.50,0.50}{##1}}}
\expandafter\def\csname PY@tok@cm\endcsname{\let\PY@it=\textit\def\PY@tc##1{\textcolor[rgb]{0.25,0.50,0.50}{##1}}}
\expandafter\def\csname PY@tok@cpf\endcsname{\let\PY@it=\textit\def\PY@tc##1{\textcolor[rgb]{0.25,0.50,0.50}{##1}}}
\expandafter\def\csname PY@tok@c1\endcsname{\let\PY@it=\textit\def\PY@tc##1{\textcolor[rgb]{0.25,0.50,0.50}{##1}}}
\expandafter\def\csname PY@tok@cs\endcsname{\let\PY@it=\textit\def\PY@tc##1{\textcolor[rgb]{0.25,0.50,0.50}{##1}}}

\def\PYZbs{\char`\\}
\def\PYZus{\char`\_}
\def\PYZob{\char`\{}
\def\PYZcb{\char`\}}
\def\PYZca{\char`\^}
\def\PYZam{\char`\&}
\def\PYZlt{\char`\<}
\def\PYZgt{\char`\>}
\def\PYZsh{\char`\#}
\def\PYZpc{\char`\%}
\def\PYZdl{\char`\$}
\def\PYZhy{\char`\-}
\def\PYZsq{\char`\'}
\def\PYZdq{\char`\"}
\def\PYZti{\char`\~}
% for compatibility with earlier versions
\def\PYZat{@}
\def\PYZlb{[}
\def\PYZrb{]}
\makeatother


    % Exact colors from NB
    \definecolor{incolor}{rgb}{0.0, 0.0, 0.5}
    \definecolor{outcolor}{rgb}{0.545, 0.0, 0.0}



    
    % Prevent overflowing lines due to hard-to-break entities
    \sloppy 
    % Setup hyperref package
    \hypersetup{
      breaklinks=true,  % so long urls are correctly broken across lines
      colorlinks=true,
      urlcolor=urlcolor,
      linkcolor=linkcolor,
      citecolor=citecolor,
      }
    % Slightly bigger margins than the latex defaults
    
    \geometry{verbose,tmargin=1in,bmargin=1in,lmargin=1in,rmargin=1in}
    
    

    \begin{document}
    
    
    \maketitle
    
    

    
    \section{Object Oriented Programming}\label{object-oriented-programming}

Object Oriented Programming (OOP) tends to be one of the major obstacles
for beginners when they are first starting to learn Python.

There are many, many tutorials and lessons covering OOP so feel free to
Google search other lessons, and I have also put some links to other
useful tutorials online at the bottom of this Notebook.

For this lesson we will construct our knowledge of OOP in Python by
building on the following topics:

\begin{itemize}
\tightlist
\item
  Objects
\item
  Using the \emph{class} keyword
\item
  Creating class attributes
\item
  Creating methods in a class
\item
  Learning about Inheritance
\item
  Learning about Polymorphism
\item
  Learning about Special Methods for classes
\end{itemize}

Lets start the lesson by remembering about the Basic Python Objects. For
example:

    \begin{Verbatim}[commandchars=\\\{\}]
{\color{incolor}In [{\color{incolor}1}]:} \PY{n}{lst} \PY{o}{=} \PY{p}{[}\PY{l+m+mi}{1}\PY{p}{,}\PY{l+m+mi}{2}\PY{p}{,}\PY{l+m+mi}{3}\PY{p}{]}
\end{Verbatim}


    Remember how we could call methods on a list?

    \begin{Verbatim}[commandchars=\\\{\}]
{\color{incolor}In [{\color{incolor}2}]:} \PY{n}{lst}\PY{o}{.}\PY{n}{count}\PY{p}{(}\PY{l+m+mi}{2}\PY{p}{)}
\end{Verbatim}


\begin{Verbatim}[commandchars=\\\{\}]
{\color{outcolor}Out[{\color{outcolor}2}]:} 1
\end{Verbatim}
            
    What we will basically be doing in this lecture is exploring how we
could create an Object type like a list. We've already learned about how
to create functions. So let's explore Objects in general:

\subsection{Objects}\label{objects}

In Python, \emph{everything is an object}. Remember from previous
lectures we can use type() to check the type of object something is:

    \begin{Verbatim}[commandchars=\\\{\}]
{\color{incolor}In [{\color{incolor}3}]:} \PY{n+nb}{print}\PY{p}{(}\PY{n+nb}{type}\PY{p}{(}\PY{l+m+mi}{1}\PY{p}{)}\PY{p}{)}
        \PY{n+nb}{print}\PY{p}{(}\PY{n+nb}{type}\PY{p}{(}\PY{p}{[}\PY{p}{]}\PY{p}{)}\PY{p}{)}
        \PY{n+nb}{print}\PY{p}{(}\PY{n+nb}{type}\PY{p}{(}\PY{p}{(}\PY{p}{)}\PY{p}{)}\PY{p}{)}
        \PY{n+nb}{print}\PY{p}{(}\PY{n+nb}{type}\PY{p}{(}\PY{p}{\PYZob{}}\PY{p}{\PYZcb{}}\PY{p}{)}\PY{p}{)}
\end{Verbatim}


    \begin{Verbatim}[commandchars=\\\{\}]
<class 'int'>
<class 'list'>
<class 'tuple'>
<class 'dict'>

    \end{Verbatim}

    So we know all these things are objects, so how can we create our own
Object types? That is where the class keyword comes in. \#\# class User
defined objects are created using the class keyword. The class is a
blueprint that defines the nature of a future object. From classes we
can construct instances. An instance is a specific object created from a
particular class. For example, above we created the object lst which was
an instance of a list object.

Let see how we can use class:

    \begin{Verbatim}[commandchars=\\\{\}]
{\color{incolor}In [{\color{incolor}4}]:} \PY{c+c1}{\PYZsh{} Create a new object type called Sample}
        \PY{k}{class} \PY{n+nc}{Sample}\PY{p}{:}
            \PY{k}{pass}
        
        \PY{c+c1}{\PYZsh{} Instance of Sample}
        \PY{n}{x} \PY{o}{=} \PY{n}{Sample}\PY{p}{(}\PY{p}{)}
        
        \PY{n+nb}{print}\PY{p}{(}\PY{n+nb}{type}\PY{p}{(}\PY{n}{x}\PY{p}{)}\PY{p}{)}
\end{Verbatim}


    \begin{Verbatim}[commandchars=\\\{\}]
<class '\_\_main\_\_.Sample'>

    \end{Verbatim}

    By convention we give classes a name that starts with a capital letter.
Note how x is now the reference to our new instance of a Sample class.
In other words, we \textbf{instantiate} the Sample class.

Inside of the class we currently just have pass. But we can define class
attributes and methods.

An \textbf{attribute} is a characteristic of an object. A
\textbf{method} is an operation we can perform with the object.

For example, we can create a class called Dog. An attribute of a dog may
be its breed or its name, while a method of a dog may be defined by a
.bark() method which returns a sound.

Let's get a better understanding of attributes through an example.

\subsection{Attributes}\label{attributes}

The syntax for creating an attribute is:

\begin{verbatim}
self.attribute = something
\end{verbatim}

There is a special method called:

\begin{verbatim}
__init__()
\end{verbatim}

This method is used to initialize the attributes of an object. For
example:

    \begin{Verbatim}[commandchars=\\\{\}]
{\color{incolor}In [{\color{incolor}5}]:} \PY{k}{class} \PY{n+nc}{Dog}\PY{p}{:}
            \PY{k}{def} \PY{n+nf}{\PYZus{}\PYZus{}init\PYZus{}\PYZus{}}\PY{p}{(}\PY{n+nb+bp}{self}\PY{p}{,}\PY{n}{breed}\PY{p}{)}\PY{p}{:}
                \PY{n+nb+bp}{self}\PY{o}{.}\PY{n}{breed} \PY{o}{=} \PY{n}{breed}
                
        \PY{n}{sam} \PY{o}{=} \PY{n}{Dog}\PY{p}{(}\PY{n}{breed}\PY{o}{=}\PY{l+s+s1}{\PYZsq{}}\PY{l+s+s1}{Lab}\PY{l+s+s1}{\PYZsq{}}\PY{p}{)}
        \PY{n}{frank} \PY{o}{=} \PY{n}{Dog}\PY{p}{(}\PY{n}{breed}\PY{o}{=}\PY{l+s+s1}{\PYZsq{}}\PY{l+s+s1}{Huskie}\PY{l+s+s1}{\PYZsq{}}\PY{p}{)}
\end{Verbatim}


    Lets break down what we have above.The special method

\begin{verbatim}
__init__() 
\end{verbatim}

is called automatically right after the object has been created:

\begin{verbatim}
def __init__(self, breed):
\end{verbatim}

Each attribute in a class definition begins with a reference to the
instance object. It is by convention named self. The breed is the
argument. The value is passed during the class instantiation.

\begin{verbatim}
 self.breed = breed
\end{verbatim}

    Now we have created two instances of the Dog class. With two breed
types, we can then access these attributes like this:

    \begin{Verbatim}[commandchars=\\\{\}]
{\color{incolor}In [{\color{incolor}6}]:} \PY{n}{sam}\PY{o}{.}\PY{n}{breed}
\end{Verbatim}


\begin{Verbatim}[commandchars=\\\{\}]
{\color{outcolor}Out[{\color{outcolor}6}]:} 'Lab'
\end{Verbatim}
            
    \begin{Verbatim}[commandchars=\\\{\}]
{\color{incolor}In [{\color{incolor}7}]:} \PY{n}{frank}\PY{o}{.}\PY{n}{breed}
\end{Verbatim}


\begin{Verbatim}[commandchars=\\\{\}]
{\color{outcolor}Out[{\color{outcolor}7}]:} 'Huskie'
\end{Verbatim}
            
    Note how we don't have any parentheses after breed; this is because it
is an attribute and doesn't take any arguments.

In Python there are also \emph{class object attributes}. These Class
Object Attributes are the same for any instance of the class. For
example, we could create the attribute \emph{species} for the Dog class.
Dogs, regardless of their breed, name, or other attributes, will always
be mammals. We apply this logic in the following manner:

    \begin{Verbatim}[commandchars=\\\{\}]
{\color{incolor}In [{\color{incolor}8}]:} \PY{k}{class} \PY{n+nc}{Dog}\PY{p}{:}
            
            \PY{c+c1}{\PYZsh{} Class Object Attribute}
            \PY{n}{species} \PY{o}{=} \PY{l+s+s1}{\PYZsq{}}\PY{l+s+s1}{mammal}\PY{l+s+s1}{\PYZsq{}}
            
            \PY{k}{def} \PY{n+nf}{\PYZus{}\PYZus{}init\PYZus{}\PYZus{}}\PY{p}{(}\PY{n+nb+bp}{self}\PY{p}{,}\PY{n}{breed}\PY{p}{,}\PY{n}{name}\PY{p}{)}\PY{p}{:}
                \PY{n+nb+bp}{self}\PY{o}{.}\PY{n}{breed} \PY{o}{=} \PY{n}{breed}
                \PY{n+nb+bp}{self}\PY{o}{.}\PY{n}{name} \PY{o}{=} \PY{n}{name}
\end{Verbatim}


    \begin{Verbatim}[commandchars=\\\{\}]
{\color{incolor}In [{\color{incolor}9}]:} \PY{n}{sam} \PY{o}{=} \PY{n}{Dog}\PY{p}{(}\PY{l+s+s1}{\PYZsq{}}\PY{l+s+s1}{Lab}\PY{l+s+s1}{\PYZsq{}}\PY{p}{,}\PY{l+s+s1}{\PYZsq{}}\PY{l+s+s1}{Sam}\PY{l+s+s1}{\PYZsq{}}\PY{p}{)}
\end{Verbatim}


    \begin{Verbatim}[commandchars=\\\{\}]
{\color{incolor}In [{\color{incolor}10}]:} \PY{n}{sam}\PY{o}{.}\PY{n}{name}
\end{Verbatim}


\begin{Verbatim}[commandchars=\\\{\}]
{\color{outcolor}Out[{\color{outcolor}10}]:} 'Sam'
\end{Verbatim}
            
    Note that the Class Object Attribute is defined outside of any methods
in the class. Also by convention, we place them first before the init.

    \begin{Verbatim}[commandchars=\\\{\}]
{\color{incolor}In [{\color{incolor}11}]:} \PY{n}{sam}\PY{o}{.}\PY{n}{species}
\end{Verbatim}


\begin{Verbatim}[commandchars=\\\{\}]
{\color{outcolor}Out[{\color{outcolor}11}]:} 'mammal'
\end{Verbatim}
            
    \subsection{Methods}\label{methods}

Methods are functions defined inside the body of a class. They are used
to perform operations with the attributes of our objects. Methods are a
key concept of the OOP paradigm. They are essential to dividing
responsibilities in programming, especially in large applications.

You can basically think of methods as functions acting on an Object that
take the Object itself into account through its \emph{self} argument.

Let's go through an example of creating a Circle class:

    \begin{Verbatim}[commandchars=\\\{\}]
{\color{incolor}In [{\color{incolor}12}]:} \PY{k}{class} \PY{n+nc}{Circle}\PY{p}{:}
             \PY{n}{pi} \PY{o}{=} \PY{l+m+mf}{3.14}
         
             \PY{c+c1}{\PYZsh{} Circle gets instantiated with a radius (default is 1)}
             \PY{k}{def} \PY{n+nf}{\PYZus{}\PYZus{}init\PYZus{}\PYZus{}}\PY{p}{(}\PY{n+nb+bp}{self}\PY{p}{,} \PY{n}{radius}\PY{o}{=}\PY{l+m+mi}{1}\PY{p}{)}\PY{p}{:}
                 \PY{n+nb+bp}{self}\PY{o}{.}\PY{n}{radius} \PY{o}{=} \PY{n}{radius} 
                 \PY{n+nb+bp}{self}\PY{o}{.}\PY{n}{area} \PY{o}{=} \PY{n}{radius} \PY{o}{*} \PY{n}{radius} \PY{o}{*} \PY{n}{Circle}\PY{o}{.}\PY{n}{pi}
         
             \PY{c+c1}{\PYZsh{} Method for resetting Radius}
             \PY{k}{def} \PY{n+nf}{setRadius}\PY{p}{(}\PY{n+nb+bp}{self}\PY{p}{,} \PY{n}{new\PYZus{}radius}\PY{p}{)}\PY{p}{:}
                 \PY{n+nb+bp}{self}\PY{o}{.}\PY{n}{radius} \PY{o}{=} \PY{n}{new\PYZus{}radius}
                 \PY{n+nb+bp}{self}\PY{o}{.}\PY{n}{area} \PY{o}{=} \PY{n}{new\PYZus{}radius} \PY{o}{*} \PY{n}{new\PYZus{}radius} \PY{o}{*} \PY{n+nb+bp}{self}\PY{o}{.}\PY{n}{pi}
         
             \PY{c+c1}{\PYZsh{} Method for getting Circumference}
             \PY{k}{def} \PY{n+nf}{getCircumference}\PY{p}{(}\PY{n+nb+bp}{self}\PY{p}{)}\PY{p}{:}
                 \PY{k}{return} \PY{n+nb+bp}{self}\PY{o}{.}\PY{n}{radius} \PY{o}{*} \PY{n+nb+bp}{self}\PY{o}{.}\PY{n}{pi} \PY{o}{*} \PY{l+m+mi}{2}
         
         
         \PY{n}{c} \PY{o}{=} \PY{n}{Circle}\PY{p}{(}\PY{p}{)}
         
         \PY{n+nb}{print}\PY{p}{(}\PY{l+s+s1}{\PYZsq{}}\PY{l+s+s1}{Radius is: }\PY{l+s+s1}{\PYZsq{}}\PY{p}{,}\PY{n}{c}\PY{o}{.}\PY{n}{radius}\PY{p}{)}
         \PY{n+nb}{print}\PY{p}{(}\PY{l+s+s1}{\PYZsq{}}\PY{l+s+s1}{Area is: }\PY{l+s+s1}{\PYZsq{}}\PY{p}{,}\PY{n}{c}\PY{o}{.}\PY{n}{area}\PY{p}{)}
         \PY{n+nb}{print}\PY{p}{(}\PY{l+s+s1}{\PYZsq{}}\PY{l+s+s1}{Circumference is: }\PY{l+s+s1}{\PYZsq{}}\PY{p}{,}\PY{n}{c}\PY{o}{.}\PY{n}{getCircumference}\PY{p}{(}\PY{p}{)}\PY{p}{)}
\end{Verbatim}


    \begin{Verbatim}[commandchars=\\\{\}]
Radius is:  1
Area is:  3.14
Circumference is:  6.28

    \end{Verbatim}

    In the \_\_init\_\_ method above, in order to calculate the area
attribute, we had to call Circle.pi. This is because the object does not
yet have its own .pi attribute, so we call the Class Object Attribute pi
instead. In the setRadius method, however, we'll be working with an
existing Circle object that does have its own pi attribute. Here we can
use either Circle.pi or self.pi. Now let's change the radius and see how
that affects our Circle object:

    \begin{Verbatim}[commandchars=\\\{\}]
{\color{incolor}In [{\color{incolor}13}]:} \PY{n}{c}\PY{o}{.}\PY{n}{setRadius}\PY{p}{(}\PY{l+m+mi}{2}\PY{p}{)}
         
         \PY{n+nb}{print}\PY{p}{(}\PY{l+s+s1}{\PYZsq{}}\PY{l+s+s1}{Radius is: }\PY{l+s+s1}{\PYZsq{}}\PY{p}{,}\PY{n}{c}\PY{o}{.}\PY{n}{radius}\PY{p}{)}
         \PY{n+nb}{print}\PY{p}{(}\PY{l+s+s1}{\PYZsq{}}\PY{l+s+s1}{Area is: }\PY{l+s+s1}{\PYZsq{}}\PY{p}{,}\PY{n}{c}\PY{o}{.}\PY{n}{area}\PY{p}{)}
         \PY{n+nb}{print}\PY{p}{(}\PY{l+s+s1}{\PYZsq{}}\PY{l+s+s1}{Circumference is: }\PY{l+s+s1}{\PYZsq{}}\PY{p}{,}\PY{n}{c}\PY{o}{.}\PY{n}{getCircumference}\PY{p}{(}\PY{p}{)}\PY{p}{)}
\end{Verbatim}


    \begin{Verbatim}[commandchars=\\\{\}]
Radius is:  2
Area is:  12.56
Circumference is:  12.56

    \end{Verbatim}

    Great! Notice how we used self. notation to reference attributes of the
class within the method calls. Review how the code above works and try
creating your own method.

\subsection{Inheritance}\label{inheritance}

Inheritance is a way to form new classes using classes that have already
been defined. The newly formed classes are called derived classes, the
classes that we derive from are called base classes. Important benefits
of inheritance are code reuse and reduction of complexity of a program.
The derived classes (descendants) override or extend the functionality
of base classes (ancestors).

Let's see an example by incorporating our previous work on the Dog
class:

    \begin{Verbatim}[commandchars=\\\{\}]
{\color{incolor}In [{\color{incolor}14}]:} \PY{k}{class} \PY{n+nc}{Animal}\PY{p}{:}
             \PY{k}{def} \PY{n+nf}{\PYZus{}\PYZus{}init\PYZus{}\PYZus{}}\PY{p}{(}\PY{n+nb+bp}{self}\PY{p}{)}\PY{p}{:}
                 \PY{n+nb}{print}\PY{p}{(}\PY{l+s+s2}{\PYZdq{}}\PY{l+s+s2}{Animal created}\PY{l+s+s2}{\PYZdq{}}\PY{p}{)}
         
             \PY{k}{def} \PY{n+nf}{whoAmI}\PY{p}{(}\PY{n+nb+bp}{self}\PY{p}{)}\PY{p}{:}
                 \PY{n+nb}{print}\PY{p}{(}\PY{l+s+s2}{\PYZdq{}}\PY{l+s+s2}{Animal}\PY{l+s+s2}{\PYZdq{}}\PY{p}{)}
         
             \PY{k}{def} \PY{n+nf}{eat}\PY{p}{(}\PY{n+nb+bp}{self}\PY{p}{)}\PY{p}{:}
                 \PY{n+nb}{print}\PY{p}{(}\PY{l+s+s2}{\PYZdq{}}\PY{l+s+s2}{Eating}\PY{l+s+s2}{\PYZdq{}}\PY{p}{)}
         
         
         \PY{k}{class} \PY{n+nc}{Dog}\PY{p}{(}\PY{n}{Animal}\PY{p}{)}\PY{p}{:}
             \PY{k}{def} \PY{n+nf}{\PYZus{}\PYZus{}init\PYZus{}\PYZus{}}\PY{p}{(}\PY{n+nb+bp}{self}\PY{p}{)}\PY{p}{:}
                 \PY{n}{Animal}\PY{o}{.}\PY{n+nf+fm}{\PYZus{}\PYZus{}init\PYZus{}\PYZus{}}\PY{p}{(}\PY{n+nb+bp}{self}\PY{p}{)}
                 \PY{n+nb}{print}\PY{p}{(}\PY{l+s+s2}{\PYZdq{}}\PY{l+s+s2}{Dog created}\PY{l+s+s2}{\PYZdq{}}\PY{p}{)}
         
             \PY{k}{def} \PY{n+nf}{whoAmI}\PY{p}{(}\PY{n+nb+bp}{self}\PY{p}{)}\PY{p}{:}
                 \PY{n+nb}{print}\PY{p}{(}\PY{l+s+s2}{\PYZdq{}}\PY{l+s+s2}{Dog}\PY{l+s+s2}{\PYZdq{}}\PY{p}{)}
         
             \PY{k}{def} \PY{n+nf}{bark}\PY{p}{(}\PY{n+nb+bp}{self}\PY{p}{)}\PY{p}{:}
                 \PY{n+nb}{print}\PY{p}{(}\PY{l+s+s2}{\PYZdq{}}\PY{l+s+s2}{Woof!}\PY{l+s+s2}{\PYZdq{}}\PY{p}{)}
\end{Verbatim}


    \begin{Verbatim}[commandchars=\\\{\}]
{\color{incolor}In [{\color{incolor}15}]:} \PY{n}{d} \PY{o}{=} \PY{n}{Dog}\PY{p}{(}\PY{p}{)}
\end{Verbatim}


    \begin{Verbatim}[commandchars=\\\{\}]
Animal created
Dog created

    \end{Verbatim}

    \begin{Verbatim}[commandchars=\\\{\}]
{\color{incolor}In [{\color{incolor}16}]:} \PY{n}{d}\PY{o}{.}\PY{n}{whoAmI}\PY{p}{(}\PY{p}{)}
\end{Verbatim}


    \begin{Verbatim}[commandchars=\\\{\}]
Dog

    \end{Verbatim}

    \begin{Verbatim}[commandchars=\\\{\}]
{\color{incolor}In [{\color{incolor}17}]:} \PY{n}{d}\PY{o}{.}\PY{n}{eat}\PY{p}{(}\PY{p}{)}
\end{Verbatim}


    \begin{Verbatim}[commandchars=\\\{\}]
Eating

    \end{Verbatim}

    \begin{Verbatim}[commandchars=\\\{\}]
{\color{incolor}In [{\color{incolor}18}]:} \PY{n}{d}\PY{o}{.}\PY{n}{bark}\PY{p}{(}\PY{p}{)}
\end{Verbatim}


    \begin{Verbatim}[commandchars=\\\{\}]
Woof!

    \end{Verbatim}

    In this example, we have two classes: Animal and Dog. The Animal is the
base class, the Dog is the derived class.

The derived class inherits the functionality of the base class.

\begin{itemize}
\tightlist
\item
  It is shown by the eat() method.
\end{itemize}

The derived class modifies existing behavior of the base class.

\begin{itemize}
\tightlist
\item
  shown by the whoAmI() method.
\end{itemize}

Finally, the derived class extends the functionality of the base class,
by defining a new bark() method.

    \subsection{Polymorphism}\label{polymorphism}

We've learned that while functions can take in different arguments,
methods belong to the objects they act on. In Python,
\emph{polymorphism} refers to the way in which different object classes
can share the same method name, and those methods can be called from the
same place even though a variety of different objects might be passed
in. The best way to explain this is by example:

    \begin{Verbatim}[commandchars=\\\{\}]
{\color{incolor}In [{\color{incolor}19}]:} \PY{k}{class} \PY{n+nc}{Dog}\PY{p}{:}
             \PY{k}{def} \PY{n+nf}{\PYZus{}\PYZus{}init\PYZus{}\PYZus{}}\PY{p}{(}\PY{n+nb+bp}{self}\PY{p}{,} \PY{n}{name}\PY{p}{)}\PY{p}{:}
                 \PY{n+nb+bp}{self}\PY{o}{.}\PY{n}{name} \PY{o}{=} \PY{n}{name}
         
             \PY{k}{def} \PY{n+nf}{speak}\PY{p}{(}\PY{n+nb+bp}{self}\PY{p}{)}\PY{p}{:}
                 \PY{k}{return} \PY{n+nb+bp}{self}\PY{o}{.}\PY{n}{name}\PY{o}{+}\PY{l+s+s1}{\PYZsq{}}\PY{l+s+s1}{ says Woof!}\PY{l+s+s1}{\PYZsq{}}
             
         \PY{k}{class} \PY{n+nc}{Cat}\PY{p}{:}
             \PY{k}{def} \PY{n+nf}{\PYZus{}\PYZus{}init\PYZus{}\PYZus{}}\PY{p}{(}\PY{n+nb+bp}{self}\PY{p}{,} \PY{n}{name}\PY{p}{)}\PY{p}{:}
                 \PY{n+nb+bp}{self}\PY{o}{.}\PY{n}{name} \PY{o}{=} \PY{n}{name}
         
             \PY{k}{def} \PY{n+nf}{speak}\PY{p}{(}\PY{n+nb+bp}{self}\PY{p}{)}\PY{p}{:}
                 \PY{k}{return} \PY{n+nb+bp}{self}\PY{o}{.}\PY{n}{name}\PY{o}{+}\PY{l+s+s1}{\PYZsq{}}\PY{l+s+s1}{ says Meow!}\PY{l+s+s1}{\PYZsq{}} 
             
         \PY{n}{niko} \PY{o}{=} \PY{n}{Dog}\PY{p}{(}\PY{l+s+s1}{\PYZsq{}}\PY{l+s+s1}{Niko}\PY{l+s+s1}{\PYZsq{}}\PY{p}{)}
         \PY{n}{felix} \PY{o}{=} \PY{n}{Cat}\PY{p}{(}\PY{l+s+s1}{\PYZsq{}}\PY{l+s+s1}{Felix}\PY{l+s+s1}{\PYZsq{}}\PY{p}{)}
         
         \PY{n+nb}{print}\PY{p}{(}\PY{n}{niko}\PY{o}{.}\PY{n}{speak}\PY{p}{(}\PY{p}{)}\PY{p}{)}
         \PY{n+nb}{print}\PY{p}{(}\PY{n}{felix}\PY{o}{.}\PY{n}{speak}\PY{p}{(}\PY{p}{)}\PY{p}{)}
\end{Verbatim}


    \begin{Verbatim}[commandchars=\\\{\}]
Niko says Woof!
Felix says Meow!

    \end{Verbatim}

    Here we have a Dog class and a Cat class, and each has a
\texttt{.speak()} method. When called, each object's \texttt{.speak()}
method returns a result unique to the object.

There a few different ways to demonstrate polymorphism. First, with a
for loop:

    \begin{Verbatim}[commandchars=\\\{\}]
{\color{incolor}In [{\color{incolor}20}]:} \PY{k}{for} \PY{n}{pet} \PY{o+ow}{in} \PY{p}{[}\PY{n}{niko}\PY{p}{,}\PY{n}{felix}\PY{p}{]}\PY{p}{:}
             \PY{n+nb}{print}\PY{p}{(}\PY{n}{pet}\PY{o}{.}\PY{n}{speak}\PY{p}{(}\PY{p}{)}\PY{p}{)}
\end{Verbatim}


    \begin{Verbatim}[commandchars=\\\{\}]
Niko says Woof!
Felix says Meow!

    \end{Verbatim}

    Another is with functions:

    \begin{Verbatim}[commandchars=\\\{\}]
{\color{incolor}In [{\color{incolor}21}]:} \PY{k}{def} \PY{n+nf}{pet\PYZus{}speak}\PY{p}{(}\PY{n}{pet}\PY{p}{)}\PY{p}{:}
             \PY{n+nb}{print}\PY{p}{(}\PY{n}{pet}\PY{o}{.}\PY{n}{speak}\PY{p}{(}\PY{p}{)}\PY{p}{)}
         
         \PY{n}{pet\PYZus{}speak}\PY{p}{(}\PY{n}{niko}\PY{p}{)}
         \PY{n}{pet\PYZus{}speak}\PY{p}{(}\PY{n}{felix}\PY{p}{)}
\end{Verbatim}


    \begin{Verbatim}[commandchars=\\\{\}]
Niko says Woof!
Felix says Meow!

    \end{Verbatim}

    In both cases we were able to pass in different object types, and we
obtained object-specific results from the same mechanism.

A more common practice is to use abstract classes and inheritance. An
abstract class is one that never expects to be instantiated. For
example, we will never have an Animal object, only Dog and Cat objects,
although Dogs and Cats are derived from Animals:

    \begin{Verbatim}[commandchars=\\\{\}]
{\color{incolor}In [{\color{incolor}22}]:} \PY{k}{class} \PY{n+nc}{Animal}\PY{p}{:}
             \PY{k}{def} \PY{n+nf}{\PYZus{}\PYZus{}init\PYZus{}\PYZus{}}\PY{p}{(}\PY{n+nb+bp}{self}\PY{p}{,} \PY{n}{name}\PY{p}{)}\PY{p}{:}    \PY{c+c1}{\PYZsh{} Constructor of the class}
                 \PY{n+nb+bp}{self}\PY{o}{.}\PY{n}{name} \PY{o}{=} \PY{n}{name}
         
             \PY{k}{def} \PY{n+nf}{speak}\PY{p}{(}\PY{n+nb+bp}{self}\PY{p}{)}\PY{p}{:}              \PY{c+c1}{\PYZsh{} Abstract method, defined by convention only}
                 \PY{k}{raise} \PY{n+ne}{NotImplementedError}\PY{p}{(}\PY{l+s+s2}{\PYZdq{}}\PY{l+s+s2}{Subclass must implement abstract method}\PY{l+s+s2}{\PYZdq{}}\PY{p}{)}
         
         
         \PY{k}{class} \PY{n+nc}{Dog}\PY{p}{(}\PY{n}{Animal}\PY{p}{)}\PY{p}{:}
             
             \PY{k}{def} \PY{n+nf}{speak}\PY{p}{(}\PY{n+nb+bp}{self}\PY{p}{)}\PY{p}{:}
                 \PY{k}{return} \PY{n+nb+bp}{self}\PY{o}{.}\PY{n}{name}\PY{o}{+}\PY{l+s+s1}{\PYZsq{}}\PY{l+s+s1}{ says Woof!}\PY{l+s+s1}{\PYZsq{}}
             
         \PY{k}{class} \PY{n+nc}{Cat}\PY{p}{(}\PY{n}{Animal}\PY{p}{)}\PY{p}{:}
         
             \PY{k}{def} \PY{n+nf}{speak}\PY{p}{(}\PY{n+nb+bp}{self}\PY{p}{)}\PY{p}{:}
                 \PY{k}{return} \PY{n+nb+bp}{self}\PY{o}{.}\PY{n}{name}\PY{o}{+}\PY{l+s+s1}{\PYZsq{}}\PY{l+s+s1}{ says Meow!}\PY{l+s+s1}{\PYZsq{}}
             
         \PY{n}{fido} \PY{o}{=} \PY{n}{Dog}\PY{p}{(}\PY{l+s+s1}{\PYZsq{}}\PY{l+s+s1}{Fido}\PY{l+s+s1}{\PYZsq{}}\PY{p}{)}
         \PY{n}{isis} \PY{o}{=} \PY{n}{Cat}\PY{p}{(}\PY{l+s+s1}{\PYZsq{}}\PY{l+s+s1}{Isis}\PY{l+s+s1}{\PYZsq{}}\PY{p}{)}
         
         \PY{n+nb}{print}\PY{p}{(}\PY{n}{fido}\PY{o}{.}\PY{n}{speak}\PY{p}{(}\PY{p}{)}\PY{p}{)}
         \PY{n+nb}{print}\PY{p}{(}\PY{n}{isis}\PY{o}{.}\PY{n}{speak}\PY{p}{(}\PY{p}{)}\PY{p}{)}
\end{Verbatim}


    \begin{Verbatim}[commandchars=\\\{\}]
Fido says Woof!
Isis says Meow!

    \end{Verbatim}

    Real life examples of polymorphism include: * opening different file
types - different tools are needed to display Word, pdf and Excel files
* adding different objects - the \texttt{+} operator performs arithmetic
and concatenation

    \subsection{Special Methods}\label{special-methods}

Finally let's go over special methods. Classes in Python can implement
certain operations with special method names. These methods are not
actually called directly but by Python specific language syntax. For
example let's create a Book class:

    \begin{Verbatim}[commandchars=\\\{\}]
{\color{incolor}In [{\color{incolor}23}]:} \PY{k}{class} \PY{n+nc}{Book}\PY{p}{:}
             \PY{k}{def} \PY{n+nf}{\PYZus{}\PYZus{}init\PYZus{}\PYZus{}}\PY{p}{(}\PY{n+nb+bp}{self}\PY{p}{,} \PY{n}{title}\PY{p}{,} \PY{n}{author}\PY{p}{,} \PY{n}{pages}\PY{p}{)}\PY{p}{:}
                 \PY{n+nb}{print}\PY{p}{(}\PY{l+s+s2}{\PYZdq{}}\PY{l+s+s2}{A book is created}\PY{l+s+s2}{\PYZdq{}}\PY{p}{)}
                 \PY{n+nb+bp}{self}\PY{o}{.}\PY{n}{title} \PY{o}{=} \PY{n}{title}
                 \PY{n+nb+bp}{self}\PY{o}{.}\PY{n}{author} \PY{o}{=} \PY{n}{author}
                 \PY{n+nb+bp}{self}\PY{o}{.}\PY{n}{pages} \PY{o}{=} \PY{n}{pages}
         
             \PY{k}{def} \PY{n+nf}{\PYZus{}\PYZus{}str\PYZus{}\PYZus{}}\PY{p}{(}\PY{n+nb+bp}{self}\PY{p}{)}\PY{p}{:}
                 \PY{k}{return} \PY{l+s+s2}{\PYZdq{}}\PY{l+s+s2}{Title: }\PY{l+s+si}{\PYZpc{}s}\PY{l+s+s2}{, author: }\PY{l+s+si}{\PYZpc{}s}\PY{l+s+s2}{, pages: }\PY{l+s+si}{\PYZpc{}s}\PY{l+s+s2}{\PYZdq{}} \PY{o}{\PYZpc{}}\PY{p}{(}\PY{n+nb+bp}{self}\PY{o}{.}\PY{n}{title}\PY{p}{,} \PY{n+nb+bp}{self}\PY{o}{.}\PY{n}{author}\PY{p}{,} \PY{n+nb+bp}{self}\PY{o}{.}\PY{n}{pages}\PY{p}{)}
         
             \PY{k}{def} \PY{n+nf}{\PYZus{}\PYZus{}len\PYZus{}\PYZus{}}\PY{p}{(}\PY{n+nb+bp}{self}\PY{p}{)}\PY{p}{:}
                 \PY{k}{return} \PY{n+nb+bp}{self}\PY{o}{.}\PY{n}{pages}
         
             \PY{k}{def} \PY{n+nf}{\PYZus{}\PYZus{}del\PYZus{}\PYZus{}}\PY{p}{(}\PY{n+nb+bp}{self}\PY{p}{)}\PY{p}{:}
                 \PY{n+nb}{print}\PY{p}{(}\PY{l+s+s2}{\PYZdq{}}\PY{l+s+s2}{A book is destroyed}\PY{l+s+s2}{\PYZdq{}}\PY{p}{)}
\end{Verbatim}


    \begin{Verbatim}[commandchars=\\\{\}]
{\color{incolor}In [{\color{incolor}24}]:} \PY{n}{book} \PY{o}{=} \PY{n}{Book}\PY{p}{(}\PY{l+s+s2}{\PYZdq{}}\PY{l+s+s2}{Python Rocks!}\PY{l+s+s2}{\PYZdq{}}\PY{p}{,} \PY{l+s+s2}{\PYZdq{}}\PY{l+s+s2}{Jose Portilla}\PY{l+s+s2}{\PYZdq{}}\PY{p}{,} \PY{l+m+mi}{159}\PY{p}{)}
         
         \PY{c+c1}{\PYZsh{}Special Methods}
         \PY{n+nb}{print}\PY{p}{(}\PY{n}{book}\PY{p}{)}
         \PY{n+nb}{print}\PY{p}{(}\PY{n+nb}{len}\PY{p}{(}\PY{n}{book}\PY{p}{)}\PY{p}{)}
         \PY{k}{del} \PY{n}{book}
\end{Verbatim}


    \begin{Verbatim}[commandchars=\\\{\}]
A book is created
Title: Python Rocks!, author: Jose Portilla, pages: 159
159
A book is destroyed

    \end{Verbatim}

    \begin{verbatim}
The __init__(), __str__(), __len__() and __del__() methods
\end{verbatim}

These special methods are defined by their use of underscores. They
allow us to use Python specific functions on objects created through our
class.

\textbf{Great! After this lecture you should have a basic understanding
of how to create your own objects with class in Python. You will be
utilizing this heavily in your next milestone project!}

For more great resources on this topic, check out:

\href{https://jeffknupp.com/blog/2014/06/18/improve-your-python-python-classes-and-object-oriented-programming/}{Jeff
Knupp's Post}

\href{https://developer.mozilla.org/en-US/Learn/Python/Quickly_Learn_Object_Oriented_Programming}{Mozilla's
Post}

\href{http://www.tutorialspoint.com/python/python_classes_objects.htm}{Tutorial's
Point}

\href{https://docs.python.org/3/tutorial/classes.html}{Official
Documentation}


    % Add a bibliography block to the postdoc
    
    
    
    \end{document}
